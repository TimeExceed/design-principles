\documentclass[UTF8,lualatex]{ctexbeamer}
\usepackage{graphicx}
\usepackage{minted}
\usepackage{tabu}
\usepackage{amsmath}

\setbeamertemplate{note page}[plain]
% $BEAMER_NOTES

\title{\kaishu 陶大讲软件工程}
  \author{陶大}
  \institute{YITU tech}
  \date{\small
    Created in July.\ 2019,\\
    Updated in May.\ 2021}

\newcommand{\pagequote}[2]{
  \Large
  \begin{quotation}
      #1
  \end{quotation}
  \flushright\normalsize --- {#2}
}

\renewcommand{\emph}[1]{\structure{#1}}

\begin{document}
\songti

\begin{frame}
\titlepage
\end{frame}

% \begin{frame}
%   \frametitle{Outline}
%   \tableofcontents
% \end{frame}

\section{破题}

\begin{frame}
    \pagequote{
        Large-system programming has over the past decade been such a tar pit,
        and many great and powerful beasts have thrashed violently in it...
        No one thing seems to cause the difficulty -- any particular paw can be pulled away.}
        {Frederick P. Brooks Jr.,\\``The Mythical Man-Month"}
\end{frame}

\begin{frame}
    \frametitle{工程是什么?}
    \pagequote{
        The creative application of scientific principles to \only<1-2>{design or develop}\only<3->{\textcolor{blue}{design or develop}} structures, machines, apparatus, or manufacturing processes, or works utilizing them singly or in combination;
        or to construct or operate the same with \only<1-3>{full cognizance of their design}\only<4->{\textcolor{orange}{full cognizance of their design}};
        or to \only<1-4>{forecast their behavior}\only<5->{\textcolor{red}{forecast their behavior}} under specific operating conditions;
        all as respects an \only<1>{intended function}\only<2->{\textcolor{green}{intended function}}, economics of operation and safety to life and property.}
        {The American Engineers' Council for Professional Development}
\end{frame}

\note{
    \begin{enumerate}
        \item intended function: 价值创造
        \item design or develop: 实现手段
        \item full cognizance of their design: 要看文档(如果有的话)。
            一般来讲,动手之前先看tutorial已经胜过80\%的码农了。
        \item forecast their behavior: 我一直强调,程序不是写出来通过测试就结束了。
            这个标准太低了。
            代码本质上要回答问题:在什么情况下如何行为。
    \end{enumerate}
}

\begin{frame}
    \frametitle{软件工程的差异}
    \pagequote{\noindent\center 软件是活的!}{陶大}
\end{frame}

\section{论可靠性}

\begin{frame}
    \frametitle{传统工程的可靠性理论}
    \begin{center}
        \large
        MTTF = Mean Time To Failure
    \end{center}
\end{frame}

\note{
    \begin{itemize}
        \item 传统工程的可靠性理论的基础是材料的物理老化。
            软件当然也要考虑硬盘坏道、GPU烧坏甚至紫外线翻掉内存一个bit,
            但是,一般来讲,软件作为一个逻辑的构建,是不会老化的。
    \end{itemize}
}

\begin{frame}
    \frametitle{软件工程的可靠性理论}
    \pagequote{What doesn't kill you, makes you stronger.}{Friedrich W. Nietzsche}
\end{frame}

\note{
    \begin{itemize}
        \item 软件的可靠性取决于从设计者到开发者对风险的识别。
            说人话,他们见识过多少坑、经历过多少坑决定了软件的可靠性。
        \item 软件更像生命体。
        \item BTW, 我不相信软件工程师存在什么35岁线。
            个别的情况会有,总会存在不需要可靠性的场景。
            行业整体上不可能。
    \end{itemize}
}

\begin{frame}
    \frametitle{传统工程的可靠性理论}
    \[
    F=1-(1-p)^n
    \]
    \begin{center}
        \begin{tabu}{l}
            $F$: 系统出错的概率;\\
            $p$: 每次改动出错的概率;\\
            $n$: 改动的次数。
        \end{tabu}
    \end{center}
\end{frame}

\note{
    \begin{itemize}
        \item 所以,要想得到一个稳定的系统,外行人自然的直觉是\textbf{不折腾}。
            只要不停的修bug,总会得到一个很稳的系统。
            这是对的。
        \item 然而这种看法忽略了软件最大的两项风险:
            \begin{enumerate}
                \item 对软件来讲,没人用是比挂掉更大的风险。
                \item 软件作为一个逻辑构建,其可靠性是有前提假设的。
                    比如疫情期间“扶我起来”的COBOL程序员们。
            \end{enumerate}
            而任何一个被人使用的系统,人总会不断挑战其边界。
            所以,\textbf{为了稳,你必须要折腾}。
        \item 所以,软件工程在实操上的核心问题是:
            如何低成本、低风险地折腾。
    \end{itemize}
}

\begin{frame}
    \frametitle{软件工程的可靠性理论}
    \pagequote{It takes all the running you can do, to keep in the same place.}
        {``Red Queen hypothesis" by Van Valen\\originated from Lewis Carroll}
\end{frame}

\section{论设计}

\begin{frame}
    \frametitle{传统工程的设计}
    \pagequote{Architecture is the design activity of the architect.}{Wikipedia}
\end{frame}

\note{
    软件工程最初从建筑学习了很多。
    现在回顾,基本上是条弯路。
    两个区别:
    \begin{enumerate}
        \item 建筑师不搬砖。
        \item 建筑设计一旦完成便不可变。
    \end{enumerate}
}

\begin{frame}
    \frametitle{软件的设计}
    \begin{columns}[t]
        \begin{column}{0.45\textwidth}
            \begin{block}{微观设计}
                要点在适度抽象
                \begin{itemize}
                    \item 从代码中来,回代码中去。
                    \item 不要怕改代码,不要怕改得多。但是要回答,怎么控制风险。
                    \item 多语言合力:动静结合、业务胶水与能力构件结合。
                \end{itemize}
            \end{block}
        \end{column}
        \begin{column}{0.45\textwidth}
            \begin{block}{宏观设计}
                要点在商业战略
                \begin{itemize}
                    \item 谁花钱、谁用、什么条件下用、解决什么问题
                    \item 不同的业务语言意味着不同的选择压力, esp.,\ 对多样性的要求
                \end{itemize}
            \end{block}
        \end{column}
    \end{columns}
\end{frame}

\note{
    \begin{itemize}
        \item 宏观设计、微观设计的工作范围约略对应概要设计和详细设计。
            但是指导思想不同。
        \item 微观设计是涌现出来的,不是设计出来的。
            抽象的目的是模块化,模块化的目的是专业化。
            不是为了代码复用。
            代码复用是一种架构手段;复用率是一个观察指标;但不是一个值得追求的目标。
        \item SOLID是工具,不是不可违背的天条。
            抽象的关键是\textbf{适度},适业务需求的度。
        \item 一门静态语言提供能力构件,提供高性能、低资源消耗。
            一门动态语言提供业务胶水,解决快速业务定制、动态行为。
            典型是浏览器。
            用进化论的黑话来讲,不同的语言应对不同的进化压力。
            识别不同的进化压力需要对业务的深度认知。
    \end{itemize}
}

\subsection{微观设计}

\begin{frame}
    \frametitle{微观设计}
    \begin{center}
        \includegraphics[height=0.7\textheight]{refactor.png}
    \end{center}
    \note[item]{
        这本书讲的是:微观设计怎样从代码中涌现出来。
        但是我们今天不讲这本书。
    }
\end{frame}

\begin{frame}
    \frametitle{两个OOP}
    \begin{columns}[t]
        \begin{column}{0.45\textwidth}
            \begin{block}{众多对象通过通讯连接在一起}
                \begin{itemize}
                    \item 这个方向诞生了Erlang、Akka、Go
                \end{itemize}
            \end{block}
            \note[item]{
                函数调用也是通讯。
            }
        \end{column}
        \begin{column}{0.45\textwidth}
            \begin{block}{通过类和继承安排各个对象的拓扑结构}
                \begin{itemize}
                    \item 这个方向导致柏拉图式设计
                \end{itemize}
            \end{block}
        \end{column}
    \end{columns}
\end{frame}

\begin{frame}[fragile]
    \frametitle{柏拉图主义、集合论与面向对象设计}
    \begin{block}{如何设计一个“鸟”类?}
        “鸟”要不要\mintinline{java}{fly()}方法?
        \begin{itemize}
            \item 这里有一只鸭子。鸭子是鸟。鸟会飞。所以鸭子会飞。
            \item 然而,鸵鸟不会飞、蝙蝠会飞。
        \end{itemize}
        \note[item]{
            柏拉图主义认为存在一个理念世界。
            理念世界里的东西都是完美的。
            现实世界的万物都是理念世界的不完美投射。
            类-继承体系的设计的问题正在于削不完美现实的脚适完美理念的履。
            \begin{itemize}
                \item 要么,你放一个空实现。
                    那么调用者,你不知道你调用了一个对象的一个方法会不会炸。
                \item 要么,你要做一个大而无当不知所云的抽象。
                    比如junit的\mintinline{java}{setUp}/\mintinline{java}{tearDown}。
            \end{itemize}

            其中的关键是,\textbf{完美的理念,by definition,是完美的,没有进化的余地}。
            毕其功于一役,我是不信的。
        }
        \note[item]{
            我批评的是作为建模工具的柏拉图主义面向对象设计。
            继承作为实现层面的语言工具,我不反对。
            虽然我不认为代码复用是一个值得追求的目标。
        }
    \end{block}
\end{frame}

\begin{frame}[fragile]
    \frametitle{OOP的解药}
    \begin{block}{面向接口(Trait-based)设计}
        接口体现能力,不谈理念,不套圈圈。
        \begin{itemize}
            \item \mintinline{java}{Flyable}接口。鸭子、蝙蝠实现这个接口,鸵鸟不实现这个接口。
            \item 抽象代数的思维
        \end{itemize}
        \note[item]{
            我不管你底下具体是什么,只要你具备群公理,你就是一个群,我就可以用。
        }
        \note[item]{
            在这个模式下,难题是怎么提炼接口。
            不同的抽象思路会导致不同的接口。
            这里没有对错。
        }
    \end{block}
    \begin{block}{函数式编程}
        函数式编程提供新的抽象思路,更关注数据流,而非控制流。
        \note[item]{
            这里不展开讲FP。
            参见我其他的slides。
        }
    \end{block}
\end{frame}

\begin{frame}
    \frametitle{On Architectures}
    \pagequote{To the man who only has a hammer, everything he encounters begins to look like a nail.}{Abraham Maslow}
\end{frame}

\note{
    我做技术选型,提供更丰富的抽象能力一定是重要的考虑因素。
    这就是我认为rust/c++好过go/java的原因,也是k8s好过spring cloud的原因。

    但是大家知道,我在基础设施部主持的开发项目都选go,没有选rust。
    现实决策需要因时制宜、因地制宜的权衡。
}

\subsection{宏观设计与组织结构}

\begin{frame}
    \frametitle{宏观设计与组织结构}
    \pagequote{
        Organizations which design systems\ldots~are constrained to produce designs
        which are copies of the communication structures of these organizations.}
        {M. Conway}
\end{frame}

\begin{frame}
    \frametitle{宏观设计与组织结构}
    \pagequote{宏观设计是伪装成技术决策的商业决策。}{陶大}
\end{frame}

\begin{frame}
    \frametitle{宏观设计与组织结构}
    \begin{block}{宏观设计须回答}
        \begin{enumerate}
            \item 各部分的用户、客户、价值何在?
            \item 基于前者的判断,回答选择压力何在(比如多样性)?
            \item 满足前述两者的资源模型如何?
            \item 前述所有诸款的动态如何?
        \end{enumerate}
    \end{block}
    \note[item]{
        更准确地说,是\textbf{战略路线在技术战线的体现}。
        所以,没有脱离业务的宏观设计。
    }
    \note[item]{
        宏观设计里无需避讳因人设岗。
    }
\end{frame}

\begin{frame}
    \frametitle{宏观设计与组织结构}
    \begin{corollary}
        压力类型不同的组件不宜放在一个团队内。
    \end{corollary}
    \note[item]{
        这里有团队能力模型的问题、团队文化的问题。
        最主要的问题是吸血-扯后腿的问题。
    }
    \note[item]{
        作为成本中心和作为利润中心,面对的压力类型不同。

        成本中心强调控制支出、控制风险。
        比方说IT。
        挣钱不是IT的工作,花钱才是。
        但是老板不会让乱花钱。
        所以会有一堆流程来约束花钱。
    }
\end{frame}

\section{案例试讲}
\subsection{中台和基础设施}

\begin{frame}
    \frametitle{中台}
    \begin{block}{是什么?}
        中台把多个相似业务模块从各自业务里剥离出来,集中到一个团队去。
    \end{block}
    \begin{block}<2->{根本问题}
        \begin{itemize}
            \item 怎样衡量中台团队的价值?
            \item 怎样确定中台模块的演进方向?
        \end{itemize}
    \end{block}
\end{frame}

\note{
    \begin{itemize}
        \item 中台,by definition, 就有复用的意思。
            回忆一下我对“代码复用”的评论。
        \item 这两个问题还处在战略层面。
            在我的理论体系里,大战略层面的问题是“专业化如何可能”?
        \item 作为经济学家,我认为\textbf{谈钱}是个好办法。
            \begin{itemize}
                \item 货币提供了价值度量。而且正常人类都懂。
                \item 货币提供了可粗可细的可比较性。
                    大业务-小业务、重要-紧急……这些权衡中,货币都可以是基础的工具。
                \item 基于货币,有大把的工具、方法辅助决策中台模块的演进方向(或者,应该称为商业决策)。
            \end{itemize}
            都到这个份上了,难道不就是云吗?
    \end{itemize}
}

\begin{frame}
    \pagequote{中台的逻辑推到极致就是云。}{陶大}
\end{frame}

\begin{frame}
    \frametitle{基础设施}
    \begin{corollary}
        基础设施也必须直面商业压力。
    \end{corollary}
\end{frame}

\note{
    这是说,基础设施也要有自己的商业模型。
    \begin{itemize}
        \item 可以不直接面对公司的客户,但要有自己的客户,可以是内部客户。
        \item 要能讲得清楚自己的价值创造,并且以什么成本创造了怎样的价值流。
        \item 要谈钱。
    \end{itemize}
}

\subsection{工具链}

\begin{frame}
    \frametitle{工具链}
    \begin{block}{资源模型及其动态}
        \begin{itemize}
            \item 产出端:收益人群是一线研发。
                \begin{itemize}
                    \item 庞大且分散
                    \item 每个人收益都不大
                    \item 没有决策花钱之权
                \end{itemize}
            \item 投入端:必须老司机把握方向盘
                \begin{itemize}
                    \item 设计和开发阶段需要老司机,维护阶段不需要。
                \end{itemize}
        \end{itemize}
    \end{block}
\end{frame}

\note{
    \begin{itemize}
        \item 工具链最难的问题不在开发、不在设计,甚至不在推广,而在商业模型。
        \item 产出端的情况意味着,工具链的改进很难评估价值。
        \item 投入端的动态意味着,DevOps是不适合工具链的。
            于是就需要回答:老司机的激励是什么?为什么他会好好做?
        \item 结论:一般而论,工具链不值得做。
    \end{itemize}
}

\subsection{测试}

\begin{frame}
    \pagequote{
        Software testing is an investigation conducted to provide stakeholders 
        with \emph{information about the quality} of the software product or service under test.}
        {Wikipedia}
\end{frame}

\note{
    \begin{itemize}
        \item Wikipedia的观点反对软件测试是quality assurance。
            Bug都是架构师设计出来、开发写出来的,修也是他们去修,测试在任何意义上都无法“保障质量”。
            \begin{itemize}
                \item 整个quality assurance的理论体系构建在MTTF之上,从建筑业和流水线工业品借鉴过来的。
                    软件工程早期的弯路之一。
            \end{itemize}
        \item 然而这个观点仍然是错的。
            本质上,\emph{软件作为纯逻辑的存在,其质量可以感受,不可度量}。
        \item 没有两个bug是完全一样的。如果有,那bug不在那个层面。
    \end{itemize}
}

\begin{frame}
    \pagequote{
        每个软件都有bug。只是我不知道它们在哪里。}
        {陶大}
\end{frame}

\begin{frame}
    \frametitle{测试}
    \begin{block}{是什么?}
        测试回答一个软件/系统,在指定环境下\emph{应当}如何行为。
    \end{block}
    \begin{block}{谁来干?}
        测试的职责\emph{应当分散}
        \begin{itemize}
            \item 产品
            \item 架构师
            \item 工程师
        \end{itemize}
    \end{block}
\end{frame}

\note{
    \begin{itemize}
        \item 产品。
            定义上,测试和PRD没有本质区别。
        \item 架构师。
            架构师是业务语言到技术语言的翻译官。
            那么把关键场景翻译成测试也很合理。
        \item 工程师。
            自己准备写什么自己没有个B数吗?
            单元测试尤其合理。
    \end{itemize}
}

\begin{frame}
    \pagequote{
        干掉测试团队!}
        {陶大}
\end{frame}

\subsection{统一API}

\begin{frame}
    \frametitle{统一API}
    \begin{block}{~}
        \begin{itemize}
            \item API本质是生态位的划分。
            \item 评价API好坏的标准:促进演化还是妨害演化。
            \item Adapters,很丑,但在很多时候是好的。
        \end{itemize}
    \end{block}
\end{frame}

\note{
    \begin{itemize}
        \item 从促演化还是妨演化的角度评价,OneAPI的结果是差的。
    \end{itemize}
}

\section{总结}

\begin{frame}
    \frametitle{总结}
    \begin{block}{~}
        \begin{itemize}
            \item 软件是活的
                \begin{itemize}
                    \item 可靠性
                \end{itemize}
            \item 微观设计
                \begin{itemize}
                    \item OOP, trait-based, FP
                \end{itemize}
            \item 宏观设计和组织结构
                \begin{itemize}
                    \item 战略路线在技术战线的体现
                \end{itemize}
            \item 案例
                \begin{itemize}
                    \item 中台逻辑的极致是云
                    \item 工具链一般不值得做
                    \item 不应当存在测试团队
                    \item OneAPI得其人而不得其时,惜哉
                \end{itemize}
        \end{itemize}
    \end{block}
\end{frame}

\begin{frame}
    \pagequote{What's your problem?}{Robin Li}
\end{frame}

\end{document}
